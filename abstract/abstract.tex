
\addcontentsline{toc}{chapter}{Abstract}

\begin{abstract}

Decision trees were successfully applied to the photometric variables se\'rsic index, n , which models the galactic profile as $
I(R) = I_{e} exp\{-b_{n} [(R/R_{e})^{1/n}-1]\}$, and D/T ratios, the proportion of light from the disk component to the total light of the galaxy, for the subset of 260 ATLAS$_{3D}$ galaxies for which the spin parameter $\lambda_{Re}$ had been calculated by Emsellem et al. \cite{Emsellem2011} and photometric quantities determined by Krajnovic et al\cite{Krajnovic2013}. When trained using these variables independently using the initialised parameter options, the results were promising, with success rate based on n reaching 73\%. When trained using D/T alone, the success was increased to 81\%, but this result was skewed by the large number of galaxies with no exponential disk component, where D/T$\lesssim $0.05, being classed universally as fast rotators. When both parameters were used for training, the algorithm did marginally better than with n alone, achieving a success rate of 75\%, where more variation was seen in estimating the spin of galaxies with no disk component, but did so incorrectly. The most promising results were achieved when the inbuilt sklearn function GridSearchCV was used that exhaustively searches the parameter space over a given range of values and/or options, and partitions the data into k subsets and recursively trains on k-1 subsets retaining 1 for validation of results. In so doing, the best parameters were found and the success rate increased dramatically to 93\%.

\end{abstract}
